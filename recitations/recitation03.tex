\documentclass[a4paper]{article}

\usepackage[utf8]{inputenc}
\usepackage[T1]{fontenc}
\usepackage{textcomp}
\usepackage[english]{babel}
\usepackage{amsmath, amssymb, amsthm}
\usepackage{hyperref, chngcntr}

\newenvironment{subproof}[1][\proofname]{%
  \renewcommand{\qedsymbol}{$\blacksquare$}%
  \begin{proof}[#1]%
}{%
  \end{proof}%
}

\newtheorem{theorem}{Theorem}[section]
\newtheorem{lemma}{Lemma}[section]

% figure support
\usepackage{import}
\usepackage{xifthen}
\pdfminorversion=7
\usepackage{pdfpages}
\usepackage{transparent}
\newcommand{\incfig}[1]{%
  \def\svgwidth{\columnwidth}
  \import{./figures/}{#1.pdf_tex}
}

\pdfsuppresswarningpagegroup=1

\begin{document}
	\section{Problem: Breaking a Chocolate Bar}%
	\label{sec:Problem: Breaking a Chocolate Bar}

	We are given a chocolate bar with $m \times n$ squares of chocolate, and our
	task is to divide it into $mn$ individual squares. We are only allowed to
	split one piece of chocolate at a time using a vertical or a horizontal
	break.

	For example, suppose that the chocolate bar is 2 × 2. The first split
	makes two pieces, both 2 × 1. Each of these pieces requires one more split to
	form single squares. This gives a total of three splits.

	Prove that the number of times you split the bar does not depend on the
	sequence of splits you make.

	\begin{proof}[Proof (by induction)]
		We will show that the number of "splits" (as defined above) on a $m \times
		n$ required to yield $mn$ unit pieces is independent of the sequence of
		splits made. This is immediate by the following theorem:

		\begin{theorem}
			\label{thm:P(m,n)}
			To divide a $m \times n$ chocolate bar into unit pieces, $mn - 1$ splits are required.
		\end{theorem}

		Let Theorem~\ref{thm:P(m,n)} be $P(m,n)$ which we would like to prove
		$\forall m, n \in \mathbb{Z}^{+}$. The general strategy is composed of the
		following steps:

		\begin{enumerate}
			\item \label{lem:base} Establish $P(1, 1)$
			\item \label{lem:n_induct} Establish $P(1, n) \implies P(1, n + 1)$
			\item \label{lem:m_induct} Establish $P(m, n) \implies P(m + 1, n)$
		\end{enumerate}

		\begin{lemma} $\forall n \in \mathbb{Z}^{+}.\ P(1, n)$
			\begin{subproof}[Proof (by induction)] \label{lem:1}
				Step~\ref{lem:base} and Step~\ref{lem:n_induct}
				establish $P(1, n)$.  Step~\ref{lem:base} is
				true since $(1 \cdot 1) - 1 = 0$ and the unit
				piece needs no more splits. Assume $P(1, n)
				\land P(1, n - 1) \land P(1, n - 2) \land
				\ldots \land P(1, 1)$ to set up strong
				induction. Since a $1 \times (n+1)$ piece of
				chocolate consists of a $1 \times k$ piece and
				a $1 \times (n + 1 - k)$ piece (for $1 \leq k
				\leq n$), the total number of splits is $1 + (n
				- 1) = n$.

				Thus, Step ~\ref{lem:n_induct}
				is established and $\forall n \in \mathbb{Z}^{+}.\ P(1, n)$ has been proven
				by strong induction (on $n$).
			\end{subproof}
		\end{lemma}

		To prove $P(m,n)$ by strong induction, we will use
		Lemma~\ref{lem:1} as the base case. For the inductive step, we
		assume $P(1, n) \land P(2, n) \land \ldots \land P(m,n)$ to
		establish $P(m+1,n)$, i.e., $(m+1)n - 1$ splits are required to
		divide a bar composed of $(m+1)n$ unit pieces. Because a
		$(m+1)n$ -sized bar consists of a piece with dimensions $k
		\times n$ and a piece with $(m + 1 - k) \times n$ dimensions
		(for $1 \leq k \leq m$), the number of splits required is $1 +
		(kn-1) + (((m + 1) - k)n - 1) = (m+1)n -  1$.  This proves
		Step~\ref{lem:m_induct} via strong induction on $m$.

	\end{proof}


	\section{Problem: The Temple of Forever}%
	\label{sec:Problem: The Temple of Forever}

	Each monk entering the Temple of Forever is given a bowl with 15 red
	beads and 12 green beads. Each time the Gong of Time rings, a monk must
	do one of two things:

	\begin{enumerate}
		\item \textit{Exchange}: If he has at least 3 red beads in his bowl, then he may exchange 3 red beads
		for 2 green beads.
		\item \textit{Swap}: He may replace each green bead in his bowl with a red bead and replace each
		red bead in his bowl with a green bead. That is, if he starts with $i$ red beads and
		$j$ green beads, then after he performs this operation, he will have $j$ red beads and $i$
		green beads.
	\end{enumerate}

	A monk may leave the Temple of Forever only when he has exactly 5 red beads and
	5 green beads in his bowl.  Let’s look at how we can represent this problem as
	a state machine.

	\begin{itemize}
		\item What do the states of the machine look like?
		\item Use the notation you developed above to represent the allowable transitions in the
		state machine.
		\item Expand the state machine diagram to the first three or four levels. Label the transitions
		according to the operation type (E for \textit{exchange} or S for \textit{swap}).
	\end{itemize}

	Now we’ll show that no monk can ever escape the Temple of Forever because the state
	(5, 5) violates an invariant of the Temple of Forever machine.

	\begin{theorem}
		No one ever leaves the Temple of Forever.
	\end{theorem}

	Prove this theorem by induction.

	\begin{proof}[Proof (by induction)]

		Let each state at after $n$ transitions (where a transition is
		either a \textit{exchange} or \textit{swap} as defined above),
		be described as $(i_{n}, j_{n})$ such that $i, j$ are the
		number of red and green beads in the bowl, respectively. We
		define the inductive hypothesis, $P(n) := i_{n} - j_{n}
		\not\equiv 0 \pmod{5}$, for all states reachable in $n$
		transitions where $n \in \mathbb{N}_0$.

		We will show that the desired state $(5, 5)$ is unreachable
		because $(5 - 5) \equiv 0 \pmod{5}$ and $P(n)$ is invariant
		over all reachable states (proving this point by induction below).

		\textbf{Base case}. $P(0) \iff (15 - 12) \not \equiv 0 \pmod{5}$.

		\textbf{Inductive case}. Assume $P(n)$ for the purposes of
		establishing $P(n) \implies P(n+1)$. Swapping doesn't change
		whether the difference in red and green beads in the subsequent
		step is divisible by 5 since $i_{n+1} - j_{n+1} \equiv j_{n} -
		i_{n} \not \equiv 0 \pmod{5}$, where the swap transition from
		step $n$ to $n+1$ is $i_{n+1} = j_{n},\ j_{n+1} = j_{n}$.
		Exchanging red beads for green beads results in $i_{n+1} -
		j_{n+1} = (i_{n} - j_{n}) - 5$, so $i_{n+1} - j_{n+1} \not
		\equiv 0 \pmod{5}$.

		With both base and inductive cases established, $P(n)$ is proven.
	\end{proof}

	Now let’s take a look at a different property of the Temple of Forever
	machine.

	\begin{theorem}
		There is a finite number of reachable states in the Temple of Forever machine.
	\end{theorem}

	Prove this theorem. (Hint: First find an invariant that suggests an
	upper bound on the number of reachable states. Be sure to prove the
	invariant.)

	\begin{proof}[Proof (by induction)]

		Let each state at after $n$ transitions (where a transition is
		either a \textit{exchange} or \textit{swap} as defined above),
		be described as $(i_{n}, j_{n})$ such that $i, j$ are the
		number of red and green beads in the bowl, respectively. We
		define the inductive hypothesis, $P(n) := i_{n} + j_{n}
		\le 27$, for all states reachable in $n$
		transitions where $n \in \mathbb{N}_0$.

		\textbf{Base case}. $P(0)$ is true because $(15 + 12) \leq 27$.

		\textbf{Inductive case}. Assume $P(n)$ for the purposes of
		establishing $P(n) \implies P(n+1)$. Swapping doesn't change
		the sum of red and green beads since $i_{n+1} + j_{n+1} = j_{n}
		+ i_{n}$, where the swap transition from step $n$ to $n+1$ is
		$i_{n+1} = j_{n},\ j_{n+1} = j_{n}$. If an exchange is possible
		(i.e., $i \ge 3$), exchanging red beads for green beads reduces
		the sum of beads in the bowl by 1, so $i_{n+1} + j_{n+1} =
		i_{n} + j_{n} - 1 < i_{n} + j_{n} \le 27$.

		With both base and inductive cases established, $P(n)$ is
		proven. Because the total number of beads in the bowl cannot be
		negative, the possible states $(i, j)$ satisfy $0 \le i + j \le
		27$, which means the possible states are finite.

	\end{proof}

	Inside the Temple of Forever, the Gong of Time rings on. As you may
	well imagine, the monks begin to recognize that no matter how many ways
	they try to exchange or swap their beads, they always seem to end up in
	some state they’ve already been in before! For one or two monks, this
	realization is all they need to propel them instantly into a state of
	enlightenment. For the overwhelming majority, however, this knowledge
	does nothing but weaken their resolve. They just get depressed. Taking
	note of the mental state of this second group, the Keeper of the Temple
	makes an unannounced appearance and proclaims to the group, “From now
	on, any monk who is able to visit 108 (108 being the mystical number
	that encompasses all of existence \footnote{See
	\url{http://astrologyforthesoul.com/vp/mysticalnumber108.html}. Also
	consider: 42 + 24 + 42 = 108.}) unique states will be allowed to leave the
	Temple of Forever.”

	Do the monks have any chance of leaving the Temple of Forever?

	\begin{theorem}
		 It is not possible to visit 108 unique states in the Temple of Forever machine.
	\end{theorem}

	Prove this theorem. (Hint: Consider a proof by contradiction.)

	\begin{proof}[Proof (by contradiction).]

		Consider the sequence of moves required to generate 108 unique
		states for this state machine. Because $i + j$ either stays the
		same or decreases by 1 for a swap and exchange, respectively,
		and the absorbing state occurs when $i + j < 3$, 25 exchanges
		at most are possible.

		Each swap adds a unique state exactly once (even though
		unlimited swaps are allowed), so the maximum sequence length is
		$52 = 2(25) + 2$ if a double swap occurs before the first exchange.

		Having reached a contradiction, the theorem is proved.

	\end{proof}

\end{document}
