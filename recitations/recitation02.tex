\documentclass[a4paper]{article}

\usepackage[utf8]{inputenc}
\usepackage[T1]{fontenc}
\usepackage{textcomp}
\usepackage[english]{babel}
\usepackage{amsmath, amssymb, amsthm, hyperref}

\newcommand{\CheckedBox}{\mbox{\ooalign{$\checkmark$\cr\hidewidth$\square$\hidewidth\cr}}\hspace{5pt}}

% figure support
\usepackage{import}
\usepackage{xifthen}
\pdfminorversion=7
\usepackage{pdfpages}
\usepackage{transparent}
\newcommand{\incfig}[1]{%
	\def\svgwidth{\columnwidth}
	\import{./figures/}{#1.pdf_tex}
}

\pdfsuppresswarningpagegroup=1

\begin{document}

\section{Problem: Geometric Sum}%
\label{sec:Problem: Geometric Sum}

Perhaps you encountered this classic formula in school:

\begin{equation*}
	1 + r + r^{2} + r^{3} + \ldots + r^{n} = \frac{1 - r^{n+1}}{1-r}
\end{equation*}

\noindent
First use the well-ordering principle, and then induction, to prove that this
formula is correct for all real values $r \neq 1$.

\begin{proof}[Proof (by contradiction \& the well-ordering principle)]

	Let $P$ be the statement we're trying to prove and let $C$ be the set of
	counterexamples of $P$, $C := \lbrace n \in \mathbb{N}_{0} \mid P(n) \text{
	is false}\rbrace$. For sake of setting up a contradiction, assume $|C| > 0$.
	Because $n$ ranges over non-negative integers, $C$ must have a smallest
	element by the well-ordering principle, called $n_{0}$. $n_{0} > 0$ since
	$P(0)$ is true. Let $k = n_{0} - 1$, so $P(k)$ is true (otherwise it would be
	a smaller element in $C$). We reach a contradiction where $P(n_{0})$ is true:

	\begin{align*}
	1 + r + r^{2} + r^{3} + \ldots + r^{k} &=  \frac{1 - r^{k + 1}}{1 - r}\\
		1 + r \left(1 + r + r^{2} + r^{3} + \ldots + r^{k} \right) &= 1 + r \left(\frac{1 - r^{k + 1}}{1 - r}\right)\\
		1 + \left(r + r^{2} + r^{3} + \ldots + r^{k+1}\right) &= \left(\frac{1 - r^{k + 2}}{1 - r}\right)\\
		1 + r^{2} + r^{3} + \ldots + r^{n_{0}} &= \left(\frac{1 - r^{n_{0} + 1}}{1 - r}\right)\\
	\end{align*}

	Hence, $C$ must be empty (i.e., $P$ has no counterexamples).
\end{proof}

\begin{proof}[Proof (by induction)]

	We will proceed by inducting on $n$ to show $P(n)$ is always true.

	\textbf{Base case.} $P(0)$ is true because $1 = \frac{1 - r^{(0 + 1)}}{1 - r}$.

	\textbf{Inductive case.} We need to show $P(n) \implies P(n+1)$. Assuming $P(n)$:
	\begin{align*}
	1 + r + r^{2} + r^{3} + \ldots + r^{n} &=  \frac{1 - r^{n + 1}}{1 - r}\\
	1 + r \left(1 + r + r^{2} + r^{3} + \ldots + r^{n} \right) &= 1 + r \left(\frac{1 - r^{n + 1}}{1 - r}\right)\\
	1 + r + r^{2} + r^{3} + \ldots + r^{n+1} &= \left(\frac{1 - r^{(n + 1) + 1}}{1 - r}\right)\\
	\therefore P(n) \implies P(n+1)
	\end{align*}

	With the base and inductive case proved, $P$ is true over all $n \in
\mathbb{N}_{0}$.  \end{proof}

\section{Problem: The Surveyevor}%
\label{sec:Problem: The Surveyevor}

In a new reality TV series called Surveyevor, a group of contestants is placed on a small
island. Before the series begins, each contestant agrees to have a small purple or red tattoo,
in the shape of an eye, applied to the middle of his or her forehead. In all, there are p $\geq$ 1
purple eyes and r $\geq$ 0 red eyes. However, none of the contestants knows the color of his or
her third eye, nor how many total purple and red eyes there are. Furthermore, there are no
mirrors and no one is allowed to discuss the tattoos ever. Therefore, everyone knows the
colors of everyone else’s third eye, but not their own. Good thing, because a contestant who
learns that he or she has a purple eye must leave the island at the end of the show that day,
and is therefore no longer eligible to win the \$1 million cash prize at the end of the show!

The contestants live in uneasy ignorance for several weeks. As time goes on, however,
most of them lose their fear of being exiled, adapt to island living, and even make friends
with one another. Things are going quite well for the islanders, but as you might suppose,
the television audience grows bored, and the show’s ratings plummet. When the network
threatens to cancel the series, the producer decides she needs to do something, fast: on the
next show, to the surprise of the happy islanders, the producer herself appears and convenes
a meeting. Very loudly, she proclaims, “I see that at least one person here has a purple eye.”
Assuming that all the contestants are master logicians, what happens?

\bigskip

\textbf{Solution}. On day $n = p$, all $p$ contestants with purple tattoos leave the island.

\begin{proof}[Proof (by induction)]

	Let $P(n)$ be defined as the conjunction of the following three statements:

	\begin{enumerate}
		\item \label{cond:1} If $p > n$, then no one has yet left the island on Day  $n$.
		\item \label{cond:2} If $p = n$, then all $p$ contestants with purple tattoos leave the island on Day $n$.
		\item \label{cond:3} If $p < n$, then no one leaves the island on Day $n$ because all purple tattooed contestants already left.
	\end{enumerate}

	We will proceed by inducting on $n$ to show $P(n)$ is always true.

	\textbf{Base case.} $P(1)$ is true when considering the truth of the
	comprising conditionals:
	\begin{enumerate}
		\item \CheckedBox If $p > 1$, then no one has yet left the island in Day $1$. This is true because each
			purple tattooed contestant knows of at least one other purple tattooed
			contestant, so in their mind, they could still possess either a red or
			purple tattoo.
		\item \CheckedBox If $p = 1$, then all $p$ contestants with purple tattoos
			leave the island on Day $1$. This is true because the sole purple tattoo bearer
			would see that all the other contestants have red tattoos and would know
			he/she has a purple tattoo.
		\item \CheckedBox If $p < 1$, then no one leaves the island on Day $1$. This is vacuously true since $p \geq 1$.
\end{enumerate}

	\textbf{Inductive case.} Assuming $P(n)$, we must show $P(n+1)$:
	\begin{enumerate}
		\item \CheckedBox If $p > n + 1$, then no one has yet left the island on Day $n+1$. By
			definition, $p > n+1 \implies p > n$. From part \ref{cond:1} of $P(n)$,
			all the purple-eyed contestants have survived Day $n$. On day $n + 1$,
			the purple eye tattooed contestants see at least $n + 1$ other
			purple-eyed peers. Seeing this many purple-eyed peers and knowing no one
			has left, a purple-eyed contestant may still think they have a red eye
			tattoo.
		\item \CheckedBox If $p = n + 1$, then all $p$ contestants with purple
			tattoos leave the island on Day $n+1$. All contestants survive Day $n$, but a
			purple-eyed contestant seeing only $n$ other purple-eyed contestants
			realizes they have a purple eye, too.
		\item \CheckedBox If $p < n + 1$, then no one leaves the island on Day $n + 1$. In the
			case where $p = n$, then everyone already left on the prior day, Day $n$.
			If $p < n$, then all the purple-eyed contestants had already left on a
			day prior to Day $n$. 
\end{enumerate}


	With the base and inductive cases proven, $P(n)$ is true over all $n \in
	\mathbb{Z}^{+}$.

\end{proof}

\end{document}
