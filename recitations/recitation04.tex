\documentclass[a4paper]{article}

\usepackage[utf8]{inputenc}
\usepackage[T1]{fontenc}
\usepackage{textcomp}
\usepackage[english]{babel}
\usepackage{amsmath, amssymb, amsthm}

\newenvironment{subproof}[1][\proofname]{%
  \renewcommand{\qedsymbol}{$\blacksquare$}%
  \begin{proof}[#1]%
}{%
  \end{proof}%
}

\newtheorem{theorem}{Theorem}[section]
\newtheorem{lemma}{Lemma}[section]

\usepackage{enumitem}

% figure support
\usepackage{import}
\usepackage{xifthen}
\pdfminorversion=7
\usepackage{pdfpages}
\usepackage{transparent}
\newcommand{\incfig}[1]{%
	\def\svgwidth{\columnwidth}
	\import{./figures/}{#1.pdf_tex}
}

\pdfsuppresswarningpagegroup=1

\begin{document}
	\section{Problem: The Pulverizer!}%
	\label{sec:Problem: The Pulverizer!}
	
	There is a pond. Inside the pond there are $n$ pebbles, arranged in a cycle. A
	frog is sitting on one of the pebbles. Whenever he jumps, he lands exactly $k$
	pebbles away in the clockwise direction, where $0 < k < n$. The frog’s meal, a
	delicious worm, lies on the pebble right next to his, in the clockwise
	direction.

	\begin{enumerate}[label=\textbf{(\alph*)}]
		\item Describe a situation where the frog can't reach the worm.

			A multiplicative inverse of $k$ under modulo $n$ doesn't exist $\iff$ the
			greatest common divisor of $k$ and $n$ is greater than 1 $\iff$ $k$ and $n$ are not co-prime.

		\item In a situation where the frog can actually reach the worm, explain
			how to use the Pulverizer to find how many jumps the frog will need.

			The Pulverizer produces the greatest common divisor of $a$ and $b$ as a
			linear combination $sa + tb = 1$. To find the multiplicative inverse
			(i.e., the number of jumps required by the frog to get to the worm), find
			$t$ such that $t \cdot b \equiv 1 \pmod{a}$ and $t > 0$.

		\item Compute the number of jumps if $n = 50$ and $k = 21$. Anything strange? Can you fix it?

			The number of jumps using the Pulverizer algorithm yields a
			multiplicative inverse of $t = -19$ (see below).  To find this value in $\{0, 1, 2,
			3, \ldots, 49 \}$, we add multiples of 50 until we yield a number in this
			set: $-19 + 50 = 31$.

			\begin{table}[h!]
			\centering
			\begin{tabular}[c]{ c c r c l }
				$x$ & $y$ & rem($x$, $y$) &  & $x - qy$ \\
				\hline
				$50$ & $21$ & $8$ & $=$ & $50 - (2)21$ \\
				$21$ & $8$ & $5$ & $=$ & $(-2)50 + (5)21$ \\
				$8$ & $5$ & $3$ & $=$ & $(3)50 - (7)21$ \\
				$5$ & $3$ & $2$ & $=$ & $(-5)50 + (12)21$ \\
				$3$ & $2$ & \textcircled{1} & $=$ & $(8)50 - (19)21$ \\
				$2$ & $1$ & $0$ & $=$ & \ldots \\
			\end{tabular}
			\end{table}
\end{enumerate}

\section{Problem: The Fibonacci Numbers}%
\label{sec:Problem: The Fibonacci Numbers}
	The Fibonacci numbers are defined as follows:

	\begin{table}[h!]
	\centering
	\begin{tabular}{c c c}
		$F_{0} = 0$ & $F_{1} = 1$ & $F_{n} = F_{n-1} + F_{n-2}$ (for $n \geq 2$)
	\end{tabular}
	\end{table}

	Give an inductive proof that the Fibonacci numbers $F_{n}$ and $F_{n+1}$ are
	relatively prime for all $n \geq 0$.

	\begin{proof}[Proof (by induction).]

		We will complete this proof by induction by inducting on $n$ with the
		inductive hypothesis $P(n) := F_{n},\ F_{n-1}$ are relatively prime
		over all nonnegative integers $n$.

		\textbf{Base cases}. $P(0)$ and $P(1)$ are true because $0, 1$ and $1, 1$ are coprime, respectively.

		\textbf{Inductive case}. To establish $P(n) \implies P(n+1)$ for integers
		$n \geq 1$, we will first prove the following lemma:

		\begin{lemma}
			\label{lem:coprime_sum}
			The sum of two coprime nonnegative integers $(i + j)$ is coprime with both $i$ and $j$.		

			\begin{subproof}[Proof (by contradiction).]

				Assume for the sake of setting up a contradiction that $i + j$ is not
				relatively prime to $i$ or not relatively prime to $j$. First, let $k >
				1$ be a divisor that divides $i$ and $i + j$. This implies $k \mid j$,
				but this results in a contradiction because $i$ and $j$ are coprime
				(i.e., no $k$ exists that divides both $i$ and $j$). Similar reasoning
				also shows that $i + j$ and $j$ are coprime.

			\end{subproof}

		\end{lemma}

		For $n \geq 1$, $P(n) \implies P(n+1)$ is true because $F_{n+1}$ and
		$F_{n}$ are coprime by Lemma \ref{lem:coprime_sum}.
			
		By induction, for all nonnegative integers $n$, $P(n)$ is proven true.

	\end{proof}

\end{document}
